%------------------------- Carga de paquetes ---------------------------
%
% Si no necesit�s alg�n paquete, comentalo.
%
% Definici�n del tama�o de p�gina y los m�rgenes:
%
\usepackage[a4paper,headheight=16pt,scale={0.7,0.8},hoffset=0.5cm]{geometry}


% Para escribir en castellano:
%
\usepackage[spanish]{babel}


%\usepackage{lstlang06}

%
% Si se prefiere tipograf�a Helvetica (Arial), descomentar las siguientes dos l�neas ( Las f�rmulas seguiran estando en Times)
%
%\usepackage{helvet}
%\renewcommand\familydefault{\sfdefault}
\usepackage{siunitx}
%
%------------------------- Carga de paquetes ---------------------------
%
% Si no necesit�s alg�n paquete, comentalo.
%
% Definici�n del tama�o de p�gina y los m�rgenes:
%
%\usepackage[a4paper,headheight=16pt,scale={0.7,0.8},hoffset=0.5cm]{geometry}

% Para escribir en castellano:
%
\usepackage[spanish]{babel}
%\usepackage[latin1]{inputenc}
%\usepackage{lstlang06}

%
% Si se prefiere tipograf�a Helvetica (Arial), descomentar las siguientes dos l�neas ( Las f�rmulas seguiran estando en Times)
%
%\usepackage{helvet}
%\renewcommand\familydefault{\sfdefault}
\usepackage{siunitx}
%
% El paquete amsmath agrega algunas funcionalidades extra a las f�rmulas. 
% Adem�s defino la numeraci�n de las tablas y figuras al estilo "Figura 2.3", en lugar de "Figura 7". (Por lo tanto, aunque no uses f�rmulas, si quer�s este tipo de numeraci�n dej� el paquete amsmath descomentado).
%
\usepackage{amsmath}
\usepackage{amsfonts} 
\usepackage{amssymb} 
\usepackage{fancybox} 
\numberwithin{equation}{section}
\numberwithin{figure}{section}
\numberwithin{table}{section}
\usepackage{listings}

%
% Para tener cabecera y pie de p�gina con un estilo personalizado:
%
\usepackage{fancyhdr}

%
% Para poner el texto "Figura X" en negrita:
% (Si no ten�s el paquete 'caption2', prob� con 'caption').
%
\usepackage[hang,bf]{caption}

%
% Para poder usar subfiguras: (al estilo Figura 2.3(b) )
%
\usepackage{subcaption}
%
% Para poder agregar notas al pie en tablas:
%
\usepackage{threeparttable}
\usepackage{multirow}

%------------------------------ graphicx ----------------------------------
%
% Para incluir im�genes, el siguiente c�digo carga el paquete graphicx 
% seg�n se est� generando un archivo dvi o un pdf (con pdflatex). 

% Para generar dvi, descoment� la linea siguiente:
%\usepackage[dvips]{graphicx}

% Para generar pdf, descoment� las dos lineas seguientes:
\usepackage[pdftex]{graphicx}
\pdfcompresslevel=9
\usepackage{float}
\usepackage{mathrsfs}
\usepackage{texdraw}
\usepackage{pdfpages}

%
%------------------------------ graphicx ----------------------------------

% Necesit�s este paquete si haces los diagramas de flujo en el programa Dia 
%\usepackage{tikz}
%\usepackage{standalone}
%\usepackage[all]{xy}

%\usepackage{pstricks,pst-node,pst-circ,pst-plot,pst-3dplot,pst-all}

% Hago que en la cabecera de p�gina se muestre a la derecha la secci�n,
% y en el pie, en n�mero de p�gina a la derecha:
%

%----------------------Encabezado y pie de página---------------------------%
\pagestyle{fancy}
%\renewcommand{\sectionmark}[1]{\markboth{}{\thesection\ \ #1}}

%Encabezado izquierdo
\lhead{\miAsignatura}
%Encabezado derecho
\rhead{\includegraphics[scale=0.2]{./Logos/Logo_FIUBA_2}}
%Pie de p�gina izquierdo
\lfoot{\miCuatri}
%Pie de p�gina central
\cfoot{}
%Pie de p�gina derecho
\rfoot{Página:\thepage}
%L�nea de la nota de pie
\renewcommand{\footrulewidth}{0.4pt}