% En el siguiente archivo se configuran las variables del trabajo práctico
%% \providecommand es similar a \newcommnad, salvo que el primero ante un 
%% conflicto en la compilación, es ignorado.

% Al comienzo de un TP se debe modificar los argumentos de los comandos

\providecommand{\miTitulo}{Tp de simulación I}
\providecommand{\miSubtitulo}{Repetidor Analógico vs Repetidor Digital}

\providecommand{\miAsignatura}{Procesos Estocásticos (86.09)}

\providecommand{\miAlumnoUno}{Alcaraz, Gonzalo}
\providecommand{\miMailAlumnoUno}{g.alcaraz@outlook.com}
\providecommand{\miPadronUno}{93874}

\providecommand{\miAlumnoDos}{Chaparro, Raúl Antonii}
\providecommand{\miMailAlumnoDos}{chaparroraulantonio@gmail.com	}
\providecommand{\miPadronDos}{96222}

\providecommand{\miAlumnoTres}{Alcaraz, Gonzalo}
\providecommand{\miMailAlumnoTres}{g.alcaraz@outlook.com}
\providecommand{\miPadronTres}{93874}

\providecommand{\miAlumnoCuatro}{Apellido, Nombre}
\providecommand{\miMailAlumnoCuatro}{Mail}
\providecommand{\miPadronCuatro}{Padron}

\providecommand{\miNumGrupo}{3}

% No es necesario modificar este
%\providecommand{\myHeaderLogo}{header_fiuba}

\providecommand{\miAutores}{Alcaraz & Chaparro}
\providecommand{\miCuatri}{Año 2017 - 2\textsuperscript{do} Cuatrimestre}
\providecommand{\miFecha}{18 de Octubre del 2017}