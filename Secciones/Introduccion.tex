	\indent En el siguiente trabajo práctico se analizará dos esquemas de comunicaciones, uno digital y otro analógico.En ambos casos se transmitiran símbolos que represetnan 1 bit. Si el bit es 1, el símbolo será A y si es -1, se simbolorizá -A. La probabilidad de que A se 1 o -1 es $\frac{1}{2}$.\\
	
	\indent En el \textbf{repetidor digital} el bloque con la letra \textbf{D} toma una decisión acerca del símbolo transmitido y lo retransmite a la etapa siguiente y es la última etapa donde el detector toma la desición final. La opereación matemática del detector \emph{D} se puede escribir como:
	
						\begin{equation}
							X_{i+1}=
									\begin{cases}
											A		& \quad \text{si} Y_i \geq 0 \\
											-A		& \quad \text{si} Y_i \leq 0
									\end{cases}
						\label{Eq.sis.digital}
						\end{equation}
	\indent En cambio en el \textbf{repetidor analógico} se toma una única decisión y ocurre en el receptor. En los casos intermedios, los símbolos recibidos sonb multiplicados por una ganancia para luego retransmitirlos a la siguiente estapa. Dichos simbolos se pueden representar según:
	
						\begin{equation}
							x_{I+1} = G_{i+1} \cdot Y_i \quad i=1,....,n-
							\label{Eq.sis.analogico}
						\end{equation}